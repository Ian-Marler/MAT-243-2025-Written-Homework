\documentclass{article}
\usepackage{graphicx} % Required for inserting images (ex: \includegraphics[width=\linewidth]{Python Code.png}  )
\usepackage{xcolor} % for coloring text
\usepackage{amsmath} % for displaying text in math equations better (ex: $a < 1 \text{or} a > 2$
\usepackage[hidelinks]{hyperref} % for using hyperlinks | 'hidelinks' removes any formatting indicating that it is a link (other than mouse hover). Without it there is an ugly blue box around the links.
\usepackage{enumitem}

\title{ID \#0000000000 – Written Homework Week \#3}

\begin{document}

\maketitle

\begin{enumerate} 
    \item (1 point each. 6 points total) Name the argument form of the following argument:
    \begin{enumerate}
        \item \textit{Dogs eat meat. Fluffy does not eat meat. Therefore, Fluffy is not a dog.}

        \color{blue}
            Answer
        \color{black}
        
        \item \textit{I am a vegetarian. Vegetarians don’t eat meat. Therefore,I don’t eat meat.}
        
        \color{blue}
            Answer
        \color{black}
        
        \item \textit{On weekdays, Karl works. On weekends, he plays computer games. Therefore, Karl always works or plays computer games.}
        
        \color{blue}
            Answer
        \color{black}
        
        \item \textit{Cat lovers spend a lot of money on cat food. I don’t spend a lot of money on cat food. Therefore, I am not a cat lover.}
        
        \color{blue}
            Answer
        \color{black}
        
        \item \textit{If we increase our space budget, we can build a moon base. If we can build a moon base, we can build a Mars base. Therefore, If we increase our space budget, we can build a Mars base.}
        
        \color{blue}
            Answer
        \color{black}
        
        \item \textit{Alex smokes. She also has a pre-existing condition. Therefore, her health insurance categorizes her as a ”smoker with a pre-existing condition”}
    
        \color{blue}
            Answer
        \color{black}
    \end{enumerate}
    
    \item (2 points) Analyze the following argument by assigning appropriate propositional variables, writing the argument formally in terms of those variables and identifying the premises and basic argument forms. Do not use predicates and quantifiers.

    \textit{Your dog plays. When she plays, she gets dirty. When she is dirty, she needs a bath. Therefore, your dog needs a bath.}
    
    \color{blue}
        Answer
    \color{black}
    
    \item (2 points) Is the following a valid argument or fallacy? \textit{If it is Sunday, then the store is closed. The store is closed. Therefore, it is Sunday.} You must explain your answer.

    \color{blue}
        Answer
    \color{black}
    
    \item Formalize the following argument by using the given predicates and then rewriting the argument as a numbered sequence of statements. Identify each statement as either a premise, or a conclusion that follows according to a rule of inference from previous statements. In that case, state the rule of inference and refer by number to the previous statements that the rule of inference used.
    \begin{enumerate}
        \setlength{\parskip}{\baselineskip}
        \item (3 points) \textit{Everybody who is at least 16 years old can get driver’s license. Azul can not have a driver’s license yet. Therefore, Azul is less than 16 years old.}
    
        Predicates: $S(x)=$” $x$ is at least $16$ years old”, $D(x)=$”$x$ can get a driver’s license”. The domain of discourse is all people.

        \color{blue}
            Answer
        \color{black}
    
        \item (4 points) \textit{Dogs bark at cats. Max is a dog. Moonbeam is a cat. Therefore, Max barks at Moonbeam}

        Predicates: $B(x,y)=$”$x$ barks at $y$”, $D(x)=$”$x$ is a dog” and $C(x)=$”\\$x$ is a cat”. The domain of discourse is all animals.
        
        \color{blue}
            Answer
        \color{black}
    
    \end{enumerate}
    
    {\setlength{\parskip}{\baselineskip}
    \item (4 points) Is the following a valid argument or fallacy? \textit{Dogs bark at cats. Lions are not dogs. Ramsey is a lion. Therefore, Ramsey does not bark at cats.} You must explain your answer. You can use the following predicates in your explanation.

    Predicates: $B(x)=$”$x$ barks at cats”, $D(x)=$”$x$ is a dog”, $L(x)$=”$x$ is a lion”. The domain of discourse is all animals.

    \color{blue}
        Answer
    \color{black}}
    
    \item (3 points) Use the rules of inference to prove the conclusion r given (all 1,2,3 and 4) the four premises listed below. Write your solution as a numbered sequence of statements. Identify each statement as either a premise, or a conclusion that follows according to a rule of inference from previous statements, or it is equivalent to a previous statement by the rules of logical equivalences. You should give the rule used by name and refer by number to the previous statement(s) that the rule was applied to.

    \begin{enumerate}[itemindent=-0.8em]
        \renewcommand{\labelenumii}{\arabic{enumii}.}
        
        \item $p\rightarrow \neg q$ (premise)
        \item $p\vee u$ (premise)
        \item $q$ (premise)
        \item $((r\wedge t)\vee p)\vee ¬u$ (premise)

        
        \color{blue}
            \item Answer
            \item Answer
            \item Answer
        \color{black}
    \end{enumerate}

\end{enumerate}


\end{document}
