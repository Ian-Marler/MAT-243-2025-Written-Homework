\documentclass{article}
\usepackage{graphicx} % Required for inserting images (ex: \includegraphics[width=\linewidth]{Python Code.png}  )
\usepackage{xcolor} % for coloring text
\usepackage{amsmath} % for displaying text in math equations better (ex: $a < 1 \text{or} a > 2$
\usepackage[hidelinks]{hyperref} % for using hyperlinks | 'hidelinks' removes any formatting indicating that it is a link (other than mouse hover). Without it there is an ugly blue box around the links.

\title{ID \#0000000000 – Written Homework Week \#2}

\begin{document}

\maketitle

\begin{enumerate}
     \item Identify each of the following statements as true of false and explain your reasoning in full sentences. The domain of discourse is the integers.
     \begin{enumerate}
        \item (2 points) $\exists x\forall y(y > x)$
        
        \color{blue}
            Answer
        \color{black}
        
        \item (2 points) $\forall y\exists x(y > x)$
        
        \color{blue}
            Answer
        \color{black}
        
        \item (2 points) $\exists x(x = 0 \rightarrow x = 1)$
        
        \color{blue}
            Answer
        \color{black}
     \end{enumerate}
     
    \item Write the formal negation of the following statements. Your negation must not contain any explicit negation symbols.
    \begin{enumerate}
        \item (2 points) $\forall x\exists y(2 < x \leq y)$
        
        \color{blue}
            Answer
        \color{black}
        
        \item (2 points) $\forall y\exists x(y > 0 \rightarrow x \leq 0)$
        
        \color{blue}
            Answer
        \color{black}
    \end{enumerate}
    
    \item Negate verbally:
        \begin{enumerate}
            \item (1 point) ”All people weigh at least 100 pounds.”
        
            \color{blue}
                Answer
            \color{black}
        
            \item (1 points) ”Somebody did not see any animals in the zoo”
        
            \color{blue}
                Answer
            \color{black}
        \end{enumerate}
        
    \item (2 points) If $P$ and $Q$ are predicates over some domain, and if it is true that $\forall x(P (x) \vee Q(x))$, must $\forall xP (x) \vee \forall xQ(x)$ also be true? Why?
    
    \color{blue}
        Answer
    \color{black}
    
    \item (2 points) Suppose $P$ is the predicate defined by $P(x, y) = x\ \text{is friends with}\ y$, where the domain of discourse is all people. (No one is considered to be friends with themselves.) \textit{Translate} the formal expression $\forall x\exists y\exists z(y \neq z \wedge P(x, y) \wedge P(x, z))$ into English.
    
    \color{blue}
        Answer
    \color{black}
    
    \item (2 points) Let $P$ be defined as in the previous problem. Is $\forall x\exists y\exists z(y \neq z \rightarrow P(x, y) \wedge P(x, z))$ true or false? Why? 
    
    \color{blue}
        Answer
    \color{black}
    
    \item Translate the following statement into a logical expression using quantifiers, logical operators and the given predicates with the given domains: 
    \\\textit{”There is a person in the class, who cannot swim”}.
    \\Let $C$ and $S$ be predicates defined by$ C(x) =$ ”$x$ is in the class” and $S(x) =$ ”$x$ can swim”.
    \begin{enumerate}
        \item (1 point) domain for $x$: students in the class.
    
        \color{blue}
            Answer
        \color{black}
        
        \item (2 points) domain for $x$: all people.
    
        \color{blue}
            Answer
        \color{black}
    \end{enumerate}
    
    \item Translate the following statement into a logical expression using quantifiers, logical operators and the given predicates with the given domains:
    \\\textit{”All students in the class have taken calculus”}.
    \\Let $C$ and $Q$ be predicates defined by $C(x) =$ ”$x$ is in the class” and
    $Q(x) =$ ”$x$ has taken calculus”.
    \begin{enumerate}
        \item (1 point) domain for $x$: students in the class.
    
        \color{blue}
            Answer
        \color{black}
        
        \item (2 points) domain for $x$: all people.
    
        \color{blue}
            Answer
        \color{black}
    \end{enumerate}
\end{enumerate}
\end{document}
